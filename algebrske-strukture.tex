\documentclass{article}

% General document formatting
\usepackage[margin=0.7in]{geometry}
\usepackage[parfill]{parskip}
\usepackage{url, hyperref}

% Related to math
\usepackage{amsmath,amssymb,amsfonts,amsthm}

% encoding and language
\usepackage{lmodern}
\usepackage[slovene]{babel}
\usepackage[utf8]{inputenc}
\usepackage[T1]{fontenc}

% multiline comments
\usepackage{verbatim}

% amogus
\usepackage{amogus}

% images
\usepackage{graphicx}
\graphicspath{ {./images/} }

% theorems
\theoremstyle{definition}
\newtheorem{definition}{Definicija}[section]
\newtheorem{lemma}{Lema}[section]
\newtheorem{conseq}{Posledica}[section]
\newtheorem{claim}{Trditev}[section]
\newtheorem{theorem}{Izrek}[section]
%%
\theoremstyle{remark}
\newtheorem*{ex}{Primer}
\newtheorem*{rem}{Opomba}

% I like my squares DARK
%\renewcommand\qedsymbol{$\blacksquare$}

% convenience purposes
\newcommand{\N}{\mathbb{N}}
\newcommand{\Z}{\mathbb{Z}}
\newcommand{\Q}{\mathbb{Q}}
\newcommand{\R}{\mathbb{R}}

% fix bordermatrix
\usepackage{etoolbox}
\let\bbordermatrix\bordermatrix
\patchcmd{\bbordermatrix}{8.75}{4.75}{}{}
\patchcmd{\bbordermatrix}{\left(}{\left[}{}{}
\patchcmd{\bbordermatrix}{\right)}{\right]}{}{}


\begin{document}	
	\title{Algebrske strukture - zapiski predavanj prof. Klav"zarja}
	\author{Yon Ploj}
	\date{2. semester 2021}
	\maketitle
	
	%\tableofcontents
	%\vspace{2cm}

	% 13. 04. 2021
	\subsection{Lastnosti operacij}
	\begin{definition}[Asociativnost]
		\[ (a \cdot b) \cdot c = a \cdot (b \cdot c) \]
	\end{definition}
	\begin{definition}[Komutativnost]
		\[ a \cdot b = b \cdot a \]
	\end{definition}
	\begin{definition}[Enota]
		\[ a \cdot e = e \cdot a = a \]
	\end{definition}
	\begin{theorem}
		Enota je enoli"cna.
	\end{theorem}
	\begin{proof}
		Predpostavimo, da obstajata dve enoti $e_1$ in $e_2$.
		Ker je $e_1$ enota, je $e_1 \cdot e_2 = e_2$.
		Ker je $e_2$ enota, je $e_1 \cdot e_2 = e_1$.
		Sledi, da je $e_1 = e_2$.
	\end{proof}

	\begin{definition}[Inverz / Obratna vrednost $a$]
		\[ a \cdot a^{-1} = a^{-1} \cdot a = e \]
	\end{definition}
	\begin{rem}
		Inverz abstraktnega mno"zenja ozna"cujemo z $a^{-1}$, inverz abstraktnega se"stevanja pa z $-a$.
	\end{rem}
	\begin{theorem}
		Inverz je enoli"cen.
	\end{theorem}
	\begin{proof}
		Predpostavimo, da obstajata dva inverza $b_1$ in $b_2$.
		\[ b_1 = b_1 \cdot e = b_1 \cdot (a \cdot b_2) = (b_1 \cdot a) \cdot b_2 = e \cdot b_2 = b_2 \]
	\end{proof}

	\section{Algebrske strukture}
	\begin{definition}[Notranja operacija mno"zice $A$]
		\[ f: A \times A \rightarrow A\]
		Z infiksno notacijo ozna"cujemo $f(a,b)$ kot $a \cdot b$ ali $ab$
	\end{definition}
	\begin{definition}[Algebrska struktura]
		Mno"zica z vsaj eno notrajno operacijo
	\end{definition}
	\begin{definition}[Grupoid]
		Mno"zica z notrajno operacijo. $(M, \cdot)$
	\end{definition}
	\begin{definition}[Polgrupa]
		Asociativen grupoid.
	\end{definition}
	\begin{definition}[Monoid]
		Polgrupa z enoto.
	\end{definition}
	\begin{definition}[Grupa]
		Monoid, kjer je vsak element obrnljiv.
	\end{definition}
	\begin{definition}[Abelova grupa]
		Komutativna grupa.
	\end{definition}

	\begin{comment}
	\begin{definition}[Kolobar]
		Mno"zica z 2 operacijama $(M, +, \cdot)$ \\
		kjer je $(M, +)$ abelova grupa in $(M, \cdot)$ monoid.
		% kokoid
	\end{definition}
	\begin{definition}[Obseg]
		Kolobar, kjer so neni"celni elementi grupa za $\cdot$
	\end{definition}
	\begin{definition}[Polje]
		Komutativni obseg
	\end{definition}
	\begin{definition}[Modul]
		Kolobar z abelovo grupo $((M, +, \cdot), (V, \oplus))$
	\end{definition}
	\begin{definition}[Vektorski prostor]
		Modul, kjer je $(M, +, \cdot)$ polje.
	\end{definition}
	\end{comment}
	\subsection{Mno"zica $\Z_n$}
	\begin{definition}[Kongruenca]
			$a$ in $b$ sta kongruentna po modulu $m$ ntk. obstajajo $p,q,r \in \Z_n$, da velja:
			\[ a = p*m + r \]
			\[ b = q*m + r \]
			\[ r < p \quad \land \quad r < q \]
	\end{definition}
	Relacija kongruence je ekvivalen"cna, zato razdeli $\Z_n$ na ekvivalen"cne razrede ostankov: $\lbrace 0, 1, \ldots, n-1 \rbrace$
	
	\begin{rem}
		V nadaljevanju bomo uporabljali operaciji $+_{n}$ in $\cdot_{n}$ kot se"stevanje/mno"zenje po modulu $n$.
	\end{rem}
	
	\begin{claim}
		$(\Z_n, +_n)$ je grupa
	\end{claim}
	\begin{claim}
		$(\Z_n, \cdot_n)$ je monoid
	\end{claim}
	$x \in \Z_n$ je obrnljiv $\iff$ $x \perp m$. Zato velja, da so vsi elementi v $\Z_p$ (kjer je $p$ pra"stevilo) obrnljivi. $\Z_p$ je torej grupa.
	
	% 20. 4. 2021 (blaze it)
	\section{Grupe}
	
\end{document}
