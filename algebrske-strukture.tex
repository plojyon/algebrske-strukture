\documentclass{article}

% General document formatting
\usepackage[margin=0.7in]{geometry}
\usepackage[parfill]{parskip}
\usepackage{url, hyperref}

% Related to math
\usepackage{amsmath,amssymb,amsfonts,amsthm}

% encoding and language
\usepackage{lmodern}
\usepackage[slovene]{babel}
\usepackage[utf8]{inputenc}
\usepackage[T1]{fontenc}

% multiline comments
\usepackage{verbatim}

% dashed underline
\usepackage{ulem}

% images
\usepackage{graphicx}
\graphicspath{ {./images/} }

% theorems
\theoremstyle{definition}
\newtheorem{definition}{Definicija}[section]
\newtheorem{lemma}{Lema}[section]
\newtheorem{conseq}{Posledica}[section]
\newtheorem{claim}{Trditev}[section]
\newtheorem{theorem}{Izrek}[section]
%%
\theoremstyle{remark}
\newtheorem*{ex}{Primer}
\newtheorem*{rem}{Opomba}

% I like my squares DARK
\renewcommand\qedsymbol{$\blacksquare$}

% convenience purposes
\newcommand{\N}{\mathbb{N}}
\newcommand{\Z}{\mathbb{Z}}
\newcommand{\Q}{\mathbb{Q}}
\newcommand{\R}{\mathbb{R}}

% fix bordermatrix
\usepackage{etoolbox}
\let\bbordermatrix\bordermatrix
\patchcmd{\bbordermatrix}{8.75}{4.75}{}{}
\patchcmd{\bbordermatrix}{\left(}{\left[}{}{}
\patchcmd{\bbordermatrix}{\right)}{\right]}{}{}


\begin{document}	
	\title{Algebrske strukture - zapiski predavanj prof. Klav"zarja}
	\author{Yon Ploj}
	\date{2. semester 2021}
	\maketitle
	
	%\tableofcontents
	%\vspace{2cm}

	% 13. 04. 2021
	\subsection{Lastnosti operacij}
	\begin{definition}[Asociativnost]
		\[ (a \cdot b) \cdot c = a \cdot (b \cdot c) \]
	\end{definition}
	\begin{definition}[Komutativnost]
		\[ a \cdot b = b \cdot a \]
	\end{definition}
	\begin{definition}[Enota]
		\[ a \cdot e = e \cdot a = a \]
	\end{definition}
	\begin{theorem}
		Enota je enoli"cna.
	\end{theorem}
	\begin{proof}
		Predpostavimo, da obstajata dve enoti $e_1$ in $e_2$.
		Ker je $e_1$ enota, je $e_1 \cdot e_2 = e_2$.
		Ker je $e_2$ enota, je $e_1 \cdot e_2 = e_1$.
		Sledi, da je $e_1 = e_2$.
	\end{proof}

	\begin{definition}[Inverz / Obratna vrednost $a$]
		\[ a \cdot a^{-1} = a^{-1} \cdot a = e \]
	\end{definition}
	\begin{rem}
		Inverz abstraktnega mno"zenja ozna"cujemo z $a^{-1}$, inverz abstraktnega se"stevanja pa z $-a$.
	\end{rem}
	\begin{theorem}
		Inverz je enoli"cen.
	\end{theorem}
	\begin{proof}
		Predpostavimo, da obstajata dva inverza $b_1$ in $b_2$.
		\[ b_1 = b_1 \cdot e = b_1 \cdot (a \cdot b_2) = (b_1 \cdot a) \cdot b_2 = e \cdot b_2 = b_2 \]
	\end{proof}

	\section{Algebrske strukture}
	\begin{definition}[Notranja operacija mno"zice $A$]
		\[ f: A \times A \rightarrow A\]
		Z infiksno notacijo ozna"cujemo $f(a,b)$ kot $a \cdot b$ ali $ab$
	\end{definition}
	\begin{definition}[Algebrska struktura]
		Mno"zica z vsaj eno notrajno operacijo
	\end{definition}
	\begin{definition}[Grupoid]
		Mno"zica z notrajno operacijo. $(M, \cdot)$
	\end{definition}
	\begin{definition}[Polgrupa]
		Asociativen grupoid.
	\end{definition}
	\begin{definition}[Monoid]
		Polgrupa z enoto.
	\end{definition}
	\begin{definition}[Grupa]
		Monoid, kjer je vsak element obrnljiv.
	\end{definition}
	\begin{definition}[Abelova grupa]
		Komutativna grupa.
	\end{definition}

	\begin{comment}
	\begin{definition}[Kolobar]
		Mno"zica z 2 operacijama $(M, +, \cdot)$ \\
		kjer je $(M, +)$ abelova grupa in $(M, \cdot)$ monoid.
		% kokoid
	\end{definition}
	\begin{definition}[Obseg]
		Kolobar, kjer so neni"celni elementi grupa za $\cdot$
	\end{definition}
	\begin{definition}[Polje]
		Komutativni obseg
	\end{definition}
	\begin{definition}[Modul]
		Kolobar z abelovo grupo $((M, +, \cdot), (V, \oplus))$
	\end{definition}
	\begin{definition}[Vektorski prostor]
		Modul, kjer je $(M, +, \cdot)$ polje.
	\end{definition}
	\end{comment}
	
	\subsection{Mno"zica $\Z_n$}
	\begin{definition}[Kongruenca]
			$a$ in $b$ sta kongruentna po modulu $m$ ntk. obstajajo $p,q,r \in \Z_n$, da velja:
			\[ a = p*m + r \]
			\[ b = q*m + r \]
			\[ r < p \quad \land \quad r < q \]
	\end{definition}
	Relacija kongruence je ekvivalen"cna, zato razdeli $\Z_n$ na ekvivalen"cne razrede ostankov: $\lbrace 0, 1, \ldots, n-1 \rbrace$
	
	\begin{rem}
		V nadaljevanju bomo uporabljali operaciji $+_{n}$ in $\cdot_{n}$ kot se"stevanje/mno"zenje po modulu $n$.
	\end{rem}
	
	\begin{claim}
		$(\Z_n, +_n)$ je grupa
	\end{claim}
	\begin{claim}
		$(\Z_n, \cdot_n)$ je monoid
	\end{claim}
	$x \in \Z_n$ je obrnljiv $\iff$ $x \perp m$. Zato velja, da so vsi elementi v $\Z_p$ (kjer je $p$ pra"stevilo) obrnljivi. $\Z_p$ je torej grupa.
	
	% 20. 4. 2021 (blaze it)
	\section{Grupe}
	\begin{definition}[Cayleyeva tabela]
		Tabela, ki prikazuje definicijo operacije v kon"cnem monoidu.
		$$
		\bbordermatrix{
			\cdot & i & r & s & x & y & z \cr
				i & i & r & s & x & y & z \cr
				r & r & s & i & y & z & x \cr
				s & s & i & r & z & x & y \cr
				x & x & z & y & i & s & r \cr
				y & y & x & z & r & i & s \cr
				z & z & y & x & s & r & i \cr
		}
		$$
	\end{definition}
	\begin{rem}
		V Cayleyevi tabeli grupe so vsi elementi v vsakem stolpcu in vsaki vrstici med seboj razli"cni (Cayleyeva tabela je latinski kvadrat reda $n$). To sledi iz izreka \ref{praviloKrajsanja}
	\end{rem}
	\begin{theorem}[Pravilo kraj"sanja]\label{praviloKrajsanja}
		"Ce je $(G, \cdot)$ grupa in $a, b, c \in G$, potem velja:
		\[ ba = ca \implies b = c \]
		\[ ab = ac \implies b = c \]
	\end{theorem}
	\begin{proof}
		Naj bo $ba = ca$. Na desni pomno"zimo z $a^{-1}$ in zaradi asociativnosti dobimo:
		\[ (ba)a^{-1} = (ca)a^{-1} \]
		\[ b(aa^{-1}) = c(aa^{-1}) \]
		\[ be = ce \]
		\[ b = c \]
	\end{proof}

	\begin{definition}[Red elementa]
		Naj bo $(G, \cdot)$ kon"cna grupa. Tedaj je red elementa $a \in G$ najmanj"se naravno "stevilo $n$, za katerega velja
		\[ a^n = e \]
	\end{definition}
	\begin{claim}
		Red elementa je dobro definiran
	\end{claim}
	\begin{proof}
		Poglejmo zaporedje: $a^1, a^2, \cdots, a^{k+1}$, kjer je $k=|G|$. Zaporedje ima $k+1$ elementov, na"sa grupa pa jih ima $k$.
		Po dirichletovem na"celu % osnovni princip kombinatorike
		\[\exists p,q: (p \neq q \land (\text{B"SS } p < q) \land a^p = a^q) \]
		Tedaj
		\[ e = (a^p)(a^p)^{-1} = (a^q)(a^p)^{-1} = a^q a^{-p} = a^{q-p} \]
		Sledi $a^{q-p} = e$, kar smo "zeleli pokazati.
	\end{proof}
	\begin{rem}
		Red enote je $1$ in ker je enota enoli"cna, je enota edini element reda $1$.
	\end{rem}
	
	\section{Podgrupe}
	\begin{definition}[Podgrupa]
		Naj bo $(G, \cdot)$ grupa. Tedaj je $H \subseteq G$ podgrupa, "ce je $(H, \cdot)$ tudi grupa. Pri tem je operacija obakrat ista. Ozna"cimo $H \leq G$.
	\end{definition}
	\begin{definition}[Prava podgrupa]
		Naj bo $(H, \cdot)$ podrgupa $(G, \cdot)$. "Ce je $H \subset G \text{ (torej } H \neq G$), je $H$ prava podgrupa $G$. Ozna"cimo $H < G$.
	\end{definition}
	
	\begin{ex}[Trivialna podgrupa]
		Za vsako grupo $G$ velja $G \leq G$ in $\lbrace e \rbrace \leq G$.
	\end{ex}
	
	\begin{ex}
		$(\Q^+, \cdot) < (\R^+, \cdot)$
	\end{ex}
	\begin{ex}
		$F := \lbrace f: \R \rightarrow \R \rbrace$. $(F, +)$ je grupa. \\
		$C := \lbrace f: \R \rightarrow \R ; f \text{ je zvezna}\rbrace$. $(C, +)$ je grupa. \\
		$(C, +) < (F, +)$
	\end{ex}

	\begin{theorem}[Glavni izrek o podgrupah]\label{glavniIzrekPodrgup}
		Naj bo $(G, \cdot)$ grupa in $\emptyset \neq H \subseteq G$. Tedaj je $(H, \cdot)$ podgrupa v $(G, \cdot)$ natanko tedaj, ko
		\[ \forall x,y \in H: (x^{-1}y \in H) \]
	\end{theorem}
	\begin{proof}
		($\Rightarrow$) Naj bosta $x,y \in H$. Ker je $(H, \cdot)$ podgrupa in s tem sama zase grupa, je tudi $x^{-1} \in H$. Zato je tudi $x^{-1}y \in H$.
		\\
		($\Leftarrow$) Naj $\forall x,y \in H: (x^{-1}y \in H)$.
		\begin{itemize}
			\item asociativnost \\
			"ce so $x,y,z \in H$, potem so tudi $x,y,z \in G$. Ker v $G$ velja asociativnost, velja tudi v $H$.
			
			\item enota \\
			Ker je $H \neq \emptyset$, $\exists x \in H$. Postavimo $y = x$. Potem je tudi $x^{-1}x = e \in H$.
			
			\item inverz \\
			Vemo, da je $e \in H$. Naj bo $x \in H$. Postavimo $y = e$: $x^{-1}y \in H \implies x^{-1}e \in H \implies x^{-1} \in H$.
			
			\item zaprtost \\
			$x, y \in H$. Vemo "ze, da je $x^{-1} \in H$, zato je tudi $(x^{-1})^{-1} \in H$. Zato je $xy = (x^{-1})^{-1}y \in H$.
		\end{itemize}
	\end{proof}
	
	Za kon"cne grupe je kriterij "se enostavnej"si:
	\begin{theorem}
		Naj bo $(G, \cdot)$ kon"cna grupa in $\emptyset \neq H \subseteq G$. Tedaj je $(H, \cdot) \leq (G, \cdot) \iff (x,y \in H \implies xy \in H)$
	\end{theorem}
	\begin{proof}
		Dokaz je tako zelo enostaven, da ga ne bomo "sli dokazovat. Glavna ideja je, da malo gledate ta zaporedja in potem dobite neke zaklju"cke. %ok boomer
	\end{proof}
	
	\begin{definition}[Cikli"cna podgrupa]
		Naj bo $(G. \cdot)$ grupa in $a \in G$. Potem naj bo
		\[ \langle a \rangle := \lbrace a^n: n \in \Z \rbrace \]
		Podgrupa $(\langle a \rangle, \cdot)$ je cikli"cna podgrupa v $G$, generirana z enoto $a$.
	\end{definition}
	\begin{claim}
		"Ce je $(G, \cdot)$ grupa in $a \in G$, potem je
		\[ (\langle a \rangle, \cdot) \leq (G, \cdot) \]
	\end{claim}
	\begin{proof}
		Ker je $a^1 = a$, je $a \in \langle a \rangle$, torej $\langle a \rangle \neq \emptyset$. Naj bosta sedaj $a^n, a^m \in \langle a \rangle$.
		Ker je \[(a^n)^{-1}a^m = (a^{-1})^na^m = a^{m-n} \in \langle a \rangle \]
		je po glavnem izreku potem $(\langle a \rangle, \cdot)$ podgrupa grupe $G$.
	\end{proof}

	\begin{ex}
		$(\Z_{12}, +_{12})$
		\\
		$\langle 3 \rangle = \lbrace 3, 6, 9, 0 \rbrace$
		\\
		$(\lbrace 0, 3, 6, 9 \rbrace, +_{12}) \leq (\Z_{12}), +_{12})$
	\end{ex}

	\begin{definition}[Center grupe]
		Naj bo $(G, \cdot)$ grupa. Potem je $Z(G)$ center grupe $G$ podmno"zica z elementi, ki komutirajo z vsemi elementi v $G$.
		\[ Z(G) = \lbrace a \in G: \forall x \in G(ax=xa) \rbrace \]
	\end{definition}

	\begin{rem}
		"Ce je $G$ abelova, je $Z(G) = G$.
	\end{rem}
	\begin{theorem}
		"Ce je $(G, \cdot)$ grupa, potem je $(Z(G), \cdot) \leq (G, \cdot)$.
	\end{theorem}
	\begin{proof}		
		Poka"zimo najprej, da $a \in Z(G) \implies a^{-1} \in Z(G)$. "Ce $a$ komutira z vsemi $x \in G$, potem tudi $a^{-1}$ komutira z vsemi $x \in G$:
		\[ a^{-1} \cdot / \quad ax = xa \quad / \cdot a^{-1} \]
		\[ a^{-1}axa^{-1} = a^{-1}xaa^{-1} \]
		\[ (a^{-1}a)xa^{-1} = a^{-1}ax(a^{-1}) \]
		\[ xa^{-1} = a^{-1}x \]
		
		Sedaj pa "se $a^{-1}b \in Z(G)$:
		\[ (a^{-1}b)x = a^{-1}(bx) = a^{-1}(xb) = (a^{-1}x)b = (xa^{-1})b = x(a^{-1}b) \]
		Po izreku \ref{glavniIzrekPodrgup} je to zadosti.
	\end{proof}

	% datum
\end{document}
